\chapter{节点和边线 Nodes and Edges}
节点就是一个路径再加上文字,路径默认为矩形,不绘制,文字也可以没有。节点本身并非路径的一部分,而是在路径创建之前或者之后被绘制出来。

\section{节点的语法}
\verb|\path| \dots\ node
\begin{math}
	\langle foreach statements \rangle [\langle options\rangle] (\langle name\rangle) at (\langle coordindate\rangle) :
	\langle animation attribute\rangle=\{\langle options\rangle\} \{\langle node contents\rangle\} \dots;
\end{math}

在\texinline{node}是后面文字内容之前的开口大括号之间都是选项内容,如果要用到\texinline{foreach},则必须紧接在\texinline{node}关键字之后。

\texinline{node}的文本可以是``脆弱''内容。即使是空文本也需要提供\texinline{{}}。

\subsection{node contents}
节点的文本内容还可以用\texinline{node contents}这个选项来设置;
\begin{texlst}
	\begin{tikzpicture}
		\path    (0,0) node [red] {A}
		         (1,0) node [blue] {B}
		         (2,0) node [green,    node contents=C]
		         (3,0) node [          node contents=D];
	\end{tikzpicture}
\end{texlst}
如果在选项里面设置文本的话,tikz在选项的\texinline{]}就结束语句了。

\subsection{at}
手册里面说\texinline{at}也可以用在选项里面,但是我没有试验成功。其实我也不需要这个功能。

\subsection{behind path}
此选项会使节点在路径之后绘制。
\begin{texlst}
	\begin{tikzpicture}
		\fill[fill=blue!50, draw=blue, very thick]
		(0,0) node[behind path, fill=red!50] {first node}
		-- (1.5, 0) node[behind path, fill=green!50] {second node}
		-- (1.5, 1) node [behind path, fill=brown!50] {third node}
		-- (0, 1) node [fill=blue!30] {fourth node} -- cycle;
	\end{tikzpicture}
\end{texlst}

\subsection{in front of path}
与\texinline{behind path}相反的作用,是\tikzname 的默认值。

\subsection{name}
节点的名称除了放在小括号里面还可以在选项里面设置。

\subsection{alias}
\texinline{alias}可以用来设置别名。

\subsection{shape}
\texinline{shape}键值可以用来设置节点的形状。

\subsection{animation}
动画暂时搞不定,先略过。
搜索网上的信息说是需要编译成SVG格式,以后再研究吧。

\subsection{every node}
\begin{texlst}
	\begin{tikzpicture}[every node/.style={draw}]
		\draw (0,0) node {A} -- (1,1) node {B};
	\end{tikzpicture}
\end{texlst}

\subsection[every shape node]{every $\langle$shape$\rangle$ node}
\begin{texlst}
	\begin{tikzpicture}[every rectangle node/.style={draw},
			every circle node/.style={draw, double}]
		\draw (0,0) node[rectangle] {A} -- (1,1) node[circle] {B};
	\end{tikzpicture}
\end{texlst}

\subsection{execute at begin/end node}
\begin{texlst}
	\begin{tikzpicture}[execute at begin node={A},
			execute at end node={D}]
		\node[execute at begin node={B}]{C};
	\end{tikzpicture}
\end{texlst}

\subsection[name prefix]{name prefix=$\langle text\rangle$}
\begin{texlst}
	\begin{tikzpicture}
		\begin{scope}[name prefix=top-]
			\node (A) at (0,1) {A};
			\node (B) at (1,1) {B};
			\draw (A) -- (B);
		\end{scope}
		\begin{scope}[name prefix=bottom-]
			\node (A) at (0,0) {A};
			\node (B) at (1,0) {B};
			\draw (A) -- (B);
		\end{scope}

		\draw [red] (top-A) -- (bottom-B);
	\end{tikzpicture}
\end{texlst}

\subsection{name suffix}
与前缀用法相同。

\section{预设的形状}
\tikzname 预设了三种形状:
\begin{itemize}
	\forcsvlist{\item}{rectangle, circle, coordindate}
\end{itemize}

\section{常用选项}
\subsection{inner sep}
\begin{texlst}
	\begin{tikzpicture}
		\draw(0,0) node[inner sep=0pt, draw] {tight}
		(0cm, 2em) node[inner sep=5pt, draw] {loose}
		(0cm, 4em) node[fill=yellow!80!black] {default};
	\end{tikzpicture}
\end{texlst}

与之类似的选项还有\forcsvlist{\texinline}{inner xsep, inner ysep, outer sep, outer xsep, outer ysep}

注意, \texinline{outer sep}在一些情况下可能会出现问题,

1. 如果一个形状被填充,但是没有描边,这时\texinline{outer sep}的值应该要为0;

2. \texinline{scale}的时候,线宽没有变化,但是\texinline{outer sep}相应的倍数变化了,所以也会有问题。

这时可以用\texinline{outer sep=auto}来解决这个问题。

\subsection{minimum height}
\subsection{minimum width}
\subsection{minimum size}
\begin{texlst}
	\begin{tikzpicture}
		\draw(0,0) node[minimum size=2cm, draw] {square};
		\draw(0,-2) node[minimum size=2cm, draw, circle] {circle};
	\end{tikzpicture}
\end{texlst}

\subsection{shape aspect}
\begin{texlst}
	\usetikzlibrary{shapes.geometric}
	\begin{tikzpicture}
		\draw (0,0) node[shape=diamond, shape aspect=1, draw] {aspect 1};
		\draw (0,-2) node[shape=diamond, shape aspect=2, draw] {aspect 2};
	\end{tikzpicture}
\end{texlst}

\subsection{shape border rotate}
\begin{texlst}
	\begin{tikzpicture}
		\node[shape border rotate=30, draw, dart, shape border uses incircle] {ABC};
	\end{tikzpicture}
\end{texlst}

\section{多部分节点}
\subsection{\textbackslash nodepart}
\begin{texlst}
	\usetikzlibrary{shapes.multipart}
	\begin{tikzpicture}[every lower node part/.style={red}]
		\node[circle split, draw, double, fill=red!20]
		{
			$q_1$
			\nodepart{lower}
			$00$
		};
	\end{tikzpicture}
\end{texlst}

\section{节点文本}
\subsection{text}
\texinline{text=}用来设置文本的颜色
\subsection{node font/font}
\texinline{node font=}或者\texinline{font=}用来设置文本的字体。

\subsection{tabular}
\begin{texlst}
	\begin{tikzpicture}
		\node [draw] {
			\begin{tabular}{cc}
				upper left & upper right \\
				lower left & lower right
			\end{tabular}
		};
	\end{tikzpicture}
\end{texlst}
\subsection[用\textbackslash\textbackslash\ 换行]{用\textbackslash\textbackslash 换行}
\begin{texlst}
	\begin{tikzpicture}[align=center]
		\node[draw] {This is a \\demonstration.};
	\end{tikzpicture}
\end{texlst}

\subsection{text width}
\begin{texlst}
	\begin{tikzpicture}
		\draw(0,0) node[fill=yellow!80!black, text width=3cm, align=right]{
			This is a demonstration text for showing how line breaking works.
		};
	\end{tikzpicture}
\end{texlst}
\subsection{对齐with flush}
\begin{texlst}
	\begin{tikzpicture}
		\draw(0,0) node[fill=yellow!80!black, text width=3cm, align=flush right]{
			This is a demonstration text for showing how line breaking works.
		};
	\end{tikzpicture}
\end{texlst}
对齐选项有\forcsvlist{\texinline}{left, right, center, flush left, flush right, flush center, justify, node}

\subsection{text height}
\begin{texlst}
	\tikz \node[draw] {y};
	\tikz \node[draw, text height=10pt] {y};
\end{texlst}

\subsection{text depth}
\begin{texlst}
	\tikz \node[draw] {y};
	\tikz \node[draw, text depth=10pt] {y};
\end{texlst}

\section{位置}
\begin{texlst}
	\begin{tikzpicture}
		\node(n) [fill=blue!20, minimum width=3cm, minimum height=2cm] {Big node};
		\draw [fill] (n.north west) circle (2pt) node[above] {north west};
		\draw [fill] (n.north) circle (2pt) node[above] {north};
		\draw [fill] (n.north east) circle (2pt) node[above] {north east};
		\draw [fill] (n.west) circle (2pt) node[left] {west};
		\draw [fill] (n.base) circle (2pt) node[below] {base};
		\draw [fill] (n.east) circle (2pt) node[right] {east};
	\end{tikzpicture}
\end{texlst}

\subsection{anchor}
\begin{texlst}
	\begin{tikzpicture}
		\draw (0,0) node [anchor=north east] {first node}
		rectangle (1,1) node[anchor=west] {second node};
	\end{tikzpicture}
\end{texlst}

\begin{texlst}
	\begin{tikzpicture}[scale=3, transform shape]
		\draw[anchor=center] (0,1) node{x} -- (0.5,1) node{y} -- (1,1) node{t};
		\draw[anchor=base] (0,0.5) node{x} -- (0.5,0.5) node{y} -- (1,0.5) node{t};
		\draw[anchor=mid] (0,0) node{x} -- (0.5,0) node{y} -- (1,0) node{t};
	\end{tikzpicture}
\end{texlst}

其他常用的基本位置有\forcsvlist{\texinline}{above, below, left, right, above left, above right, below left}\\
\forcsvlist{\texinline}{below right, centered}

\subsection{positioning 库}
可以使用尺寸的表达式:
\begin{texlst}
	\begin{tikzpicture}
		\draw[help lines] (0,0) grid (2,2);
		\node at (1,1) [above=2pt+3pt, draw] {above};
	\end{tikzpicture}
\end{texlst}
还可以使用数字,默认单位为\texinline{cm}:
\begin{texlst}
	\begin{tikzpicture}
		\draw[help lines] (0,0) grid (2,2);
		\node at (1,1) [above=.5, draw] {above};
	\end{tikzpicture}
\end{texlst}

还有一种更加复杂的形式(用above时效果看不出来):
\begin{texlst}
	\begin{tikzpicture}
		\draw[help lines] (0,0) grid (2,2);
		\node at (1,1) [above left=.2 and 4mm, draw] {above left};
	\end{tikzpicture}
\end{texlst}

相对位置使用\texinline{of}:
\begin{texlst}
	\begin{tikzpicture}[every node/.style={draw}]
		\draw [help lines] (0,0) grid (2,2);
		\node (somenode) at (1,1) {some node};

		\node [above=5mm of somenode.north east] {\tiny 5mm of somenode.north east};
		\node [above=1cm of somenode.north] {\tiny 1cm of somenode.north};
	\end{tikzpicture}
\end{texlst}

\texinline{on grid}方式排列:
\begin{texlst}
	\begin{tikzpicture}[every node/.style=draw]
		\draw [help lines] (0,0) grid (2,3);

		\node (a1) at (0,0) {not gridded};
		\node (b1) [above=1cm of a1] {fooy};
		\node (c1) [above=1cm of b1] {a};

		\node (a2) at (2,0) {gridded};
		\node (b2) [on grid, above=1cm of a2] {fooy};
		\node (c2) [on grid, above=1cm of b2] {a};

	\end{tikzpicture}
\end{texlst}

\texinline{node distance}默认为$1cm\ and\ 1cm$。
\tikzname 中的两种节点距离写法有如下区别:


\begin{enumerate}
	\item \texinline{node distance=1cm},这种写法相当于在$x$和$y$方向上各移动$\frac12\sqrt2$cm
	\item \texinline{node distance=1cm and 1cm},在$x$和$y$方向各移动1cm。
\end{enumerate}

\texinline{positioning}库提供了各种位置:

\forcsvlist{\texinline}{above, above left, above right, below, below right, below left}

\forcsvlist{\texinline}{base left, base right, mid left, mid right}

此外,还可以用\texinline{matrix}和\texinline{graphdrawing}库来进行节点的排列。

\section{fit}
\begin{texlst}
	\begin{tikzpicture}
		\node (root) {root}
		child {node(a){a}}
		child {
				node(b){b}
				child{node (d){d}}
				child{node (e){e}}
			}
		child {node (c) {c}}
		;
		\node[draw=red, inner sep=0pt, thick, ellipse, fit=(root) (b) (d) (e)] {};
		\node[draw=blue, inner sep=0pt, thick, ellipse, fit=(b) (c) (d)]{};
	\end{tikzpicture}
\end{texlst}

\section{变形}
\begin{texlst}
	\begin{tikzpicture}[every node/.style={draw}]
		\draw[help lines](0,0) grid (3,2);
		\draw (1,0) node{A}
		(2,0) node[rotate=90, scale=1.5] {B};
		\draw[rotate=30](1,0) node{A}
		(2,0) node[rotate=90, scale=1.5] {B};
		\draw[rotate=60](1,0) node[transform shape] {A}
		(2,0) node[transform shape, rotate=90, scale=1.5] {B};
	\end{tikzpicture}
\end{texlst}
上面的例子中可以看到第三组用了\texinline{transform shape}以后,在旋转的过程中节点整个的形状也跟着进行了旋转。

还有一个\texinline{transform shape nonlinear}我没有太明白,后面有机会再说吧。

\section{设置节点在直线或曲线上面的位置}
\begin{texlst}
	\begin{tikzpicture}
		\draw (0,0) -- (3,1)
		node[pos=0]{0}
		node[pos=0.5]{1/2}
		node[pos=0.9]{9/10}
		;
	\end{tikzpicture}
\end{texlst}

\begin{texlst}
	\begin{tikzpicture}
		\draw[help lines] (0,0) grid (3,2);
		\draw (2,0) arc [x radius=1, y radius=2, start angle=0, end angle=180]
		node foreach \t in {0,0.125,...,1} [pos=\t, auto] {\t};
	\end{tikzpicture}
\end{texlst}

\begin{texlst}
	\begin{tikzpicture}
		\draw (0,0) .. controls +(right:3.5cm) and +(right:3.5cm) .. (0,3)
		node foreach \p in {0,0.125,...,1} [pos=\p] {\p};
	\end{tikzpicture}
\end{texlst}

在\texinline{|\textbar|-}或\texinline{-|\textbar|}直角相交模式下,0.5正好在直角的位置。

\subsection{auto}

\begin{texlst}
	\begin{tikzpicture}[auto=left, every node/.style={circle, fill=blue!20}]
		\node (a) at (-1, -2) {a};
		\node (b) at (1, -2) {b};
		\node (c) at (2, -1) {c};
		\node (d) at (2, 1) {d};
		\node (e) at (1, 2) {e};
		\node (f) at (-1, 2) {f};
		\node (g) at (-2, 1) {g};
		\node (h) at (-2, -1) {h};

		\foreach \from/\to in {a/b, b/c, c/d, d/e, e/f, f/g, g/h, h/a}
		\draw[->] (\from) -- (\to) node[midway, fill=red!20] {\from--\to};
	\end{tikzpicture}
\end{texlst}
\texinline{auto}的值可以设为\texinline{right}
或者\texinline{left},根据设定值分别位于线条的左边或者右边。另外也可以设置为\texinline{false}进行取消。\texinline{auto}若与其他位置选项\texinline{above}等一起使用,则相当于给\texinline{auto}设置为\texinline{false}。

\texinline{swap}会将当前的自动定位反向,如果当前是\texinline{left},则会变成\texinline{right}。

\texinline{'}是\texinline{swap}的快捷方式。

\subsection{sloped}
这个选项会让节点跟着路径的方向进行旋转。但是默认情况下不让出现颠倒的文字的。

如果需要颠倒的文字,可以使用\texinline{allow upside down}选项。

另外还有几个选项用来控制节点文字离路径两端距离:

\begin{center}
	\begin{tabular}{ ll }\hline
		at start        & pos=0     \\
		very near start & pos=0.125 \\
		near start      & pos=0.25  \\
		midway          & pos=0.5   \\
		near end        & pos=0.75  \\
		very near end   & pos=0.875 \\
		at end          & pos=1     \\ \hline
	\end{tabular}
\end{center}

\section{label and pin}
\texinline{label}会使得一个新的\texinline{node}添加到当前的\texinline{node}。
\subsection{label position}
\texinline{label position}可以设置为角度的值,也可以设置为文字的方向。
下面是一些可以设置的文字方向:

\forcsvlist{\texinline}{east, south, west, north}等等

\forcsvlist{\texinline}{above, below, left, right, above left}等等

\subsection{absolute}
用下面两个示例来演示\texinline{absolute}的效果。
\begin{texlst}
	\begin{tikzpicture}[rotate=-80, every label/.style={draw, red}]
		\node[transform shape, rectangle, draw, label=right:label] {main node};
	\end{tikzpicture}
	\begin{tikzpicture}[rotate=-80, every label/.style={draw, red}, absolute]
		\node[transform shape, rectangle, draw, label=right:label] {main node};
	\end{tikzpicture}
\end{texlst}

上面的例子里面也使用了\texinline{label=right:label
	name}这样的语法。方向的地方还可以用角度。此外,还可以给\texinline{label}用上颜色。

\begin{texlst}
	\begin{tikzpicture}
		\node[transform shape, rotate=90, rectangle, draw, label={[red]center:R}]{main node};
	\end{tikzpicture}
\end{texlst}

也可以给\texinline{label}设置一个名称:
\begin{texlst}
	\begin{tikzpicture}
		\node [circle, draw, label={[name=label node]above left:$a,b$}]{};
		\draw (label node) -- +(1,1);
	\end{tikzpicture}
\end{texlst}

\subsection{label distance}
此选项用来设置主节点与\texinline{label}之间的距离。
\begin{texlst}
	\begin{tikzpicture}[label distance=5mm]
		\node[circle, draw, label=right:X,
			label=above right:Y,
			label=above:Z] {my circle};
	\end{tikzpicture}
\end{texlst}

\subsection{every label}
用来设置所有\texinline{label}的属性,默认值是\texinline{draw=none, fill=none}。

\subsection{pin}
\texinline{pin}跟\texinline{label}很像,只是多一根连线连接\texinline{pin}与主节点。
与\texinline{label}一样,\texinline{pin}也有\texinline{pin distance}、\texinline{every pin}以及\texinline{pin
	position}。此外,因为\texinline{pin}有连线,所以多了两个新的属性,\texinline{pin edge}和\\\texinline{every pin edge}。
\begin{texlst}
	\begin{tikzpicture}[pin distance=10mm]
		\node[circle, draw, pin={[pin edge={blue, thick}]right:X}, pin=above:Z] {my circle};
	\end{tikzpicture}
\end{texlst}

\begin{texlst}
	\begin{tikzpicture}[every pin edge/.style={},
			initial/.style={pin={[pin distance=5mm,
									pin edge={<-, shorten <=1pt}]left:start}}]
		\node[circle, draw, initial]{my circle};
	\end{tikzpicture}
\end{texlst}

\subsection{quotes}
\begin{texlst}
	\begin{tikzpicture}
		\node ["my label" red, draw] {my node};
	\end{tikzpicture}
\end{texlst}
\texinline{quotes}是\texinline{label}或者\texinline{pin}的便捷写法。默认情况下\texinline{quotes}生成的是\texinline{label},即\\\texinline{quotes
	mean label},当然也可以设置为\texinline{quotes mean pin}。
\begin{texlst}
	\begin{tikzpicture}[quotes mean pin]
		\node["$90^\circ$" above, "$180^\circ$" left, circle, draw] {circle};
	\end{tikzpicture}
\end{texlst}

此外,对于所有相应的\texinline{label}和\texinline{pin}可以用\texinline{every label quotes}和\texinline{every pin
	quotes}来设置样式。

可以用\texinline{node quotes mean=}来设置\texinline{label}或者\texinline{pin}的样式,如下例:
\tikzset{node quotes mean={label={[#2, name={#1}]#1}}}
\begin{texlst}
	\begin{tikzpicture}
		\node["1", "2" label position=left, circle, draw] {circle};
		\draw (1) -- (2);
	\end{tikzpicture}
\end{texlst}

\section{连接节点 -- 坐标方式}
如果给节点命名后,则可以将节点的名称用小括号括起来当作坐标一样来使用,当然也可以使用节点的各个锚点。不过,\tikzname
会聪明的处理线条的连接,只会将连线绘制到节点的边界,而不会真的绘制到节点的中心。
\begin{texlst}
	\begin{tikzpicture}
		\path    (0,0) node (x)               {Hello World!}
		         (3,1) node [circle, draw](y) {$\int_1^2 \mathrm d x$};

		\draw[->, blue] (x) -- (y);
		\draw[->, red] (x) -| node[near start, below] {label} (y);
		\draw[->, orange] (x) .. controls +(up:1cm) and +(left:1cm) .. node[above, sloped] {label} (y);
	\end{tikzpicture}
\end{texlst}

\section{连接节点 -- edge方式}
\begin{texlst}
	\begin{tikzpicture}
		\node foreach \name/\angle in {a/0, b/90, c/180, d/270}
		(\name) at (\angle:1) {$\name$};

		\path[->]
			(b) edge (a)
			    edge (c)
			    edge [dotted] (d)
			(c) edge (a)
			    edge (d)
			(d) edge (a);
	\end{tikzpicture}
\end{texlst}

\begin{texlst}
	\begin{tikzpicture}
		\node foreach \name/\angle in {a/0, b/90, c/180, d/270}
		(\name) at (\angle:1.5) {$\name$};
		\path[->]
			(b) edge                           node [above right] {$5$} (a)
			    edge (c)
			    edge [-, dotted]               node [below, sloped] {missing} (d)
			(c) edge (a)
			    edge (d)
			(d) edge [red]                     node [above, sloped] {very}
			    node [below, sloped] {bad} (a);
	\end{tikzpicture}
\end{texlst}

在\texinline{node}的\texinline{edge}连线上也同样可以使用\texinline{quotes}。
注意下例中的\texinline{'}的用法:
\begin{texlst}
	\begin{tikzpicture}
		\draw (0,0) edge ["left", "right"',
				"start" near start, "end" near end] (4,0);
	\end{tikzpicture}
\end{texlst}

\section{在外部引用节点}
\subsection{在别的图片中引用节点}
\begin{enumerate}
	\item 首先要将相关的图片设置为\texinline{remember picture}。
	\item 在需要引用节点的图片中要设置为\texinline{overlay},这个选项会将包围的盒子计算功能关闭。
	\item 需要编译两遍。
\end{enumerate}

\begin{texlst}
	\begin{tikzpicture}[remember picture]
		\node[circle, fill=red!50](n1) {};
	\end{tikzpicture}
\end{texlst}
\begin{texlst}
	\begin{tikzpicture}[remember picture]
		\node{};
		\node[rectangle, fill=blue!50](n2) at (2,0){};
	\end{tikzpicture}
\end{texlst}
\begin{texlst}
	\begin{tikzpicture}[remember picture]
		\node(c) [circle, draw] {Big circle};
		\draw[overlay, ->, very thick, red, opacity=.5] (c) to[bend left] (n1)
		(n1) -| (n2)
		(n1) -- (n2);
	\end{tikzpicture}
\end{texlst}

\subsection{引用本页上的节点}
\tikzname 提供了一个特殊的节点\texinline{current page}。
\begin{texlst}
	\begin{tikzpicture}[remember picture, overlay]
		\node [xshift=1cm, yshift=1cm] at (current page.south west) [text width=7cm, fill=red!20, rounded corners, above
			right] {This is an absolutely positioned text in the lower left corner. No shipout-hackery is used.};
	\end{tikzpicture}
\end{texlst}

\begin{texlst}
	\begin{tikzpicture}[remember picture, overlay]
		\draw [line width=1mm, opacity=0.25] (current page.center) circle(3cm);
	\end{tikzpicture}
\end{texlst}

\begin{texlst}
	\begin{tikzpicture}[remember picture, overlay]
		\node [line width=1mm, scale=10, opacity=.25, rotate=60]  at (current page.center) {Example};
	\end{tikzpicture}
\end{texlst}

\section{节点的追加代码和追加选项}
\tikzname 可以对现在节点进行追加设置。如下例:
\newpage
\begin{texlst}
	\begin{tikzpicture}
		\node [draw, circle] (a) {hello};
		\node also [label=above:world] (a);
	\end{tikzpicture}
\end{texlst}

\begin{texlst}
	\begin{tikzpicture}
		\node [draw, circle] (a) {Hello};
		\path [late options={name=a, label=right:world}];
	\end{tikzpicture}
\end{texlst}
